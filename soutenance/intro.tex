\begin{frame}
  \textbf{Probl\'ematique :} \\
  \begin{center}Comment faciliter la conception d'un logiciel informatique ?\end{center}
  \vspace{10px}
  \pause
  \begin{center}
	\begin{tabular}{rp{100px}}
	\only<2-> \textbf{Une solution :} & Langage UML\\
	\vspace{10px}
	\pause
	\only<3-> \textbf{Notre solution :} & libUML !
	  
	\end{tabular}
  \end{center}
\end{frame}
\begin{frame}
	\tableofcontents
\end{frame}
\section{Notre projet}
\subsection{Pr\'esentation des acteurs}

\begin{frame}{Pr\'esentation des acteurs}
	% A mettre sur la même ligne
	\begin{minipage}{4.6cm}
		\begin{block}{\'Etudiants}
			\pause
				Antoine de \bsc{Roquemaurel}\\
			\pause
				Mathieu \bsc{Soum}\\
			\pause
				Geoffroy \bsc{Subias}\\
			\pause
				Marie-Ly \bsc{Tang}
		\end{block}
	\end{minipage}
	\pause
	~~  
	\begin{minipage}[r]{4.4cm}
		\begin{block}{Enseignants}
			\pause
				Thierry \bsc{Millan} (\textit{Client})\\
			\pause
				Caroline \bsc{Kross} (\textit{Tutrice})
				\newline \newline
		\end{block}
	\end{minipage}
\end{frame}
\subsection{D\'etails du projet}
\begin{frame}{Le projet en 4 points}
  % - Le titre doit résumer le transparent dans un langage compréhensible
  %   par tous ceux qui ne suivront rien de ce qu'il y a sur ce transparent.
  \begin{itemize}
  \setbeamercovered{transparent}
	\item<1,2> Biblioth\`eque d'objets graphiques
	\item<1,3> UML (\textbf{U}nified \textbf{M}odeling \textbf{L}anguage)
	\item<1,4> R\'eutilisable
	\item<1,5> Java/JGraphX \& NetBeans
  \end{itemize}
\end{frame}


